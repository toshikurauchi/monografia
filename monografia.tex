\documentclass[12pt]{article}
\usepackage[brazil]{babel}
\usepackage{setspace}
\usepackage{boxedminipage}
\usepackage[latin1]{inputenc}
\usepackage{latexsym}
\usepackage{amsfonts}
\usepackage{hyperref}
\usepackage{graphicx}
\usepackage{wrapfig}
\usepackage{longtable}
\usepackage{subfigure}
\usepackage{epsfig}
\usepackage[font=small, textfont=it, labelfont=bf]{caption}
\usepackage{caption}
\usepackage{listings}
\usepackage{color}

\setlength{\paperheight}{29.7cm}
\setlength{\footskip}{1.0cm}
\setlength{\textheight}{20.0cm}
\setlength{\textwidth}{16.0cm}
\setlength{\oddsidemargin}{0.0cm}
\setlength{\topmargin}{0.0cm}

\definecolor{lbcolor}{rgb}{0.9,0.9,0.9}
\lstset{backgroundcolor=\color{lbcolor},rulecolor=}
\lstset{language=C++}
\lstset{basicstyle=\footnotesize}

\lstnewenvironment{codigo}
{\vspace{10pt}\hfill\lstset{frame=single}\minipage{0.9\textwidth}}
{\endminipage\hfill\null}

\begin{document}
\title{Pandora's Box Graphic Engine - Uma Engine Gr�fica com Aplica��o em Visualiza��o de Campos Tensoriais}
\author{Andrew Toshiaki Nakayama Kurauchi \and Victor Kendy Harada \\ Orientador: Prof. Dr. Marcel Parolin Jackowski}
\maketitle
\thispagestyle{empty}

\newpage
\setcounter{page}{1}
\tableofcontents

\section{Introdu��o}

\begin{wrapfigure}{r}{0.5\textwidth}
\vspace{-20pt}
\begin{center}
\includegraphics[width=0.5\textwidth]{lizard}
\end{center}
\vspace{-20pt}
\caption{Cena complexa~\cite{lizard} gerada a partir de um conjunto de t�cnicas avan�adas de computa��o gr�fica, dentre as quais o raytracing~\cite{raytracing}, que consiste em gerar raios de luz partindo da c�mera at� os objetos.}
\vspace{-10pt}
\end{wrapfigure}

	O r�pido desenvolvimento da tecnologia nas �ltimas d�cadas permitiu um avan�o significativo na qualidade e velocidade de gera��o de imagens de computa��o gr�fica. Entretanto ainda existem muitos desafios.

	A quantidade de informa��o a ser processada � grande se comparada a capacidade de processamento. Al�m disso imagens foto realistas podem levar dias para serem geradas. Por esse motivo � necess�ria a utiliza��o de diversas t�cnicas de otimiza��o visando produzir um resultado de melhor apar�ncia de forma eficiente (tanto com rela��o ao tempo, quanto aos recursos consumidos).

	Nesse contexto a visualiza��o cient�fica pode adicionar complexidade ao problema. Tal aplica��o de computa��o gr�fica est� muitas vezes associada a uma grande quantidade de dados complexos a serem processados e visualizados. Por outro lado existem informa��es que podem ser irrelevantes para o usu�rio desse tipo de aplica��o, tais como ilumina��o (e sombra), textura de materiais, ou at� mesmo a profundidade. Sendo assim � poss�vel utilizar outros tipos de t�cnicas que se adequem melhor a essas restri��es.

\begin{center}
\begin{longtable}{c}
\includegraphics[width=0.5\textwidth]{fountain}
\end{longtable}
\end{center}
\vspace{-40pt}
\hspace{0.25\textwidth}\parbox{0.5\textwidth}{\captionof{figure}{Imagem
resultante de uma simula��o da forma��o de uma gal�xia de disco~\cite{galaxy}. Essa simula��o inclui a visualiza��o da forma��o estelar e ventos gal�cticos, o que aumenta a complexidade da cena.}}

\subsection{Motiva��o}

	As t�cnicas de otimiza��o aplicadas no desenvolvimento de programas de computa��o gr�fica s�o recorrentes em diversos aplicativos. Por esse motivo s�o desenvolvidas engines gr�ficas como OpenSceneGraph~\cite{openscenegraph} e Ogre3D~\cite{ogre3d}, que possibilitam a utiliza��o de diversas t�cnicas previamente implementadas, assim como disponibilizam maneiras de descrever cenas complexas atrav�s de estruturas de dados conhecidas. Entretanto, tais engines evolu�ram de tal forma que o tempo de aprendizagem � longo.

\subsection{Objetivos}
	O intuito desse trabalho � desenvolver a Pandora's Box Graphics Engine, uma engine gr�fica que implemente t�cnicas b�sicas de computa��o gr�fica e que seja f�cil de ser aprendida e utilizada. Al�m disso deseja-se que ela seja customiz�vel, permitindo a sua utiliza��o em diversos tipos de aplica��o, como a gera��o de imagens foto realistas, o processamento de imagens ou a visualiza��o cient�fica. Como exemplo de visualiza��o cient�fica ser� implementado um visualizador de campos tensoriais com o aux�lio da engine.

	O estudo de tensores e campos tensoriais � de import�ncia fundamental para diversas �reas do conhecimento. S�o encontradas aplica��es no estudo de fen�menos s�smicos~\cite{Engelsma_cwp-655visualization}, estruturas eletr�nicas~\cite{electronic-structure} e imagens de resson�ncia magn�tica sens�veis a difus�o~\cite{citeulike:3787472}. A visualiza��o de tais campos �, portanto, de grande relev�ncia para o avan�o do conhecimento. Na aplica��o desenvolvida ser� utilizada para testes a visualiza��o da representa��o elipsoidal de campos tensoriais provenientes de imagens de resson�ncia magn�tica sens�veis a difus�o.


\section{Conceitos e tecnologias estudadas}
Conceitos utilizados na implementa��o da engine, assim como tensores e suas formas de visualiza��o. Al�m de uma breve introdu��o ao OpenGL e C++.

\subsection{Linguagem C++98}
A linguagem de programa��o C++ foi desenvolvida como uma evolu��o da linguagem C. %� uma linguagem estruturada e orientada a objetos

\subsubsection{Vari�veis e fun��es}
As vari�veis e fun��es s�o declaradas com a mesma sintaxe utilizada na linguagem C (um qualificador seguido de um tipo e de um nome, podendo receber um valor inicial):

% Pensar em uma forma melhor de mostrar isso.
\begin{verbatim}
<qualificador> <tipo> <nome>[=<valor inicial>]
\end{verbatim}

\subsubsection{Classes e objetos}
Classes s�o definidas pelas palavras reservadas \texttt{class} ou \texttt{struct}, sendo que a diferen�a entre elas � o fato de atributos de uma \texttt{class} serem privados por padr�o e os de uma \texttt{struct}, p�blicos. � poss�vel adicionar modificadores de acesso atrav�s das palavras \texttt{public}, \texttt{private} e \texttt{protected} que definem acesso livre em todos os escopos, acesso restrito � classe e acesso restrito � classe e suas subclasses respectivamente. Construtores s�o declarados como fun��es com o nome da classe. A declara��o de uma classe \texttt{Coordenada2d} poderia ocorrer das seguintes maneiras:

Utilizando \texttt{class}:

\begin{verbatim}
class Coordenada2d {
public:
    Coordenada2d(double x, double y);
private:
    double x,y;
};
\end{verbatim}

Utilizando \texttt{struct}:

\begin{verbatim}
struct Coordenada2d {
    double x,y;
};
\end{verbatim}

Para se instanciar um objeto existem duas op��es. A primeira consiste em declarar uma vari�vel da classe desejada (da mesma forma como vari�veis do tipo \texttt{int} s�o declaradas, por exemplo). Na declara��o da vari�vel a classe ser� instanciada e inicializada. A segunda maneira � a utiliza��o da palavra reservada \texttt{new} seguida do nome da classe. O valor devolvido � um ponteiro para uma inst�ncia da classe desejada. Seguindo o exemplo anterior, a classe Coordenada2d poderia ser instanciada das seguintes formas:

\begin{verbatim}
Coordenada2d coordenada1;
Coordenada2d * coordenada2 = new Coordenada2d;
\end{verbatim}

Os argumentos do construtor s�o passados da seguinte maneira:

\begin{verbatim}
Coordenada2d coordenada1(1.0, 2.0);
Coordenada2d * coordenada2 = new Coordenada2d(1.0, 2.0);
\end{verbatim}

\subsubsection{Namespaces}
Para evitar colis�o de identificadores como nomes de classes, vari�veis e fun��es � poss�vel utilizar um \texttt{namespace}:

\begin{verbatim}
namespace nome {
// Entidades
    int n;
}
\end{verbatim}

Onde \texttt{nome} � qualquer identificador v�lido e \texttt{Entidades} � o conjunto de classes, vari�veis e fun��es que ficar�o protegidas dentro de um \texttt{namespace}.

Para referenciar identificadores de um \texttt{namespace} � utilizada a seguinte sintaxe:

\begin{verbatim}
<nome do namespace>::<identificador>
\end{verbatim}

Para acessar a vari�vel \texttt{n} do exemplo utiliza-se o seguinte c�digo:

\begin{verbatim}
nome::n
\end{verbatim}

\subsubsection{Templates}
Possibilitam programa��o gen�rica

\subsubsection{A Standard Template Library (STL)}

A Standard Template Library (STL)~\cite{stl}� uma biblioteca C++ baseada em templates que oferece um conjunto de algoritmos e estruturas de dados configur�veis.

\subsection{Boost C++ Libraries}

Boost~\cite{boost} � uma biblioteca de C++ de c�digo aberto inicialmente desenvolvida como uma extens�o para a STL dentro do \texttt{boost}. Os principais m�dulos da boost utilizados no desenvolvimento da engine foram \texttt{bind}, \texttt{function} e \texttt{smart\_ptr}. 

O m�dulo \texttt{function} cria functors que s�o capazes de armazenar tanto fun��es nativas da linguagem ou quanto outros functors. Essas fun��es s�o criadas atrav�s da seguinte constru��o:

\begin{verbatim}
void f (int i, int j, int k) {
    // implementa��o
}
// boost::function[numero de argumentos]<tipo de retorno, tipos dos argumentos>
boost::function3<void,int,int,int> funcao = f;
\end{verbatim}

A biblioteca \texttt{bind} cria functors (classes que implementam o m�todo operator ()) a partir de uma fun��o ou outro functor mas com alguns argumentos j� posicionados, por exemplo:

\begin{verbatim}
boost::bind(f, 1, _1, _2)
\end{verbatim}

Nesse caso boost::bind cria um functor que recebe apenas dois argumentos, indicados pelos nomes \_1 e \_2, o primeiro e o segundo argumento do functor criado respectivamente.

As funcionalidades de bind e function podem ser combinadas:

\begin{verbatim}
// functor sem retorno que recebe dois argumentos inteiros
boost::function2<void, int, int> functor1 = boost::bind(f, 1, _1, _2);
// cria um boost::function a partir do resultado do boost::bind de uma 
// boost::function
boost::function1<void, int> functor2 = boost::bind(functor1, 2, _1);
\end{verbatim}

Os m�dulos bind e function da boost s�o utilizados principalmente em conjunto com algoritmos da STL, como o std::for\_each.

Os \texttt{smart\_ptr} s�o classes que guardam ponteiros para �reas de mem�ria alocadas dinamicamente. Essas classes tem comportamento parecido com o de ponteiros nativos da linguagem, com a diferen�a de que elas desalocam o recurso a que referenciam em momento adequado. Existem seis tipos smart\_ptr definidos:

\begin{itemize}
\item scoped\_ptr: representa o conceito de ponteiro n�o compartilh�vel, logo n�o podem ser copiados. Desalocam a mem�ria ao sair de escopo.
\item scoped\_array: mesmo que scoped\_ptr mas para um array de objetos.
\item shared\_ptr: representa um ponteiro que tem diversos donos, pode ser copiado e s� � destru�do quando todos os donos forem destru�dos.
\item shared\_array: mesmo que shared\_ptr mas para arrays.
\item weak\_ptr: s�o ponteiros para recursos de um shared\_ptr mas que n�o s�o donos do recurso.
\item intrusive\_ptr: s�o shared\_ptr que guardam dentro do objeto para o qual apontam o contador de refer�ncias utilizado internamente.
\end{itemize}

\subsection{OpenGL}

\subsection{Windows API}
%Citar anteriormente...
Como citado anteriormente, o OpenGL n�o � respons�vel por criar janelas e tratar eventos gerados pelo sistema operacional, esse tratamento � dependente do ambiente no qual o programa ser� executado. Nesse trabalho utilizamos apenas a API do sistema operacional Windows.

Para se inicializar um contexto para a execu��o de programas de OpenGL, � necess�rio criar uma janela, para isso usa-se a seguinte fun��o\cite{doccreatewindow}:

\begin{verbatim}
HWND WINAPI CreateWindow(
  __in_opt  LPCTSTR lpClassName,
  __in_opt  LPCTSTR lpWindowName,
  __in      DWORD dwStyle,
  __in      int x,
  __in      int y,
  __in      int nWidth,
  __in      int nHeight,
  __in_opt  HWND hWndParent,
  __in_opt  HMENU hMenu,
  __in_opt  HINSTANCE hInstance,
  __in_opt  LPVOID lpParam
);
\end{verbatim}

Onde:
\begin{itemize}
\item lpClassName: representa uma classe registrada anteriormente no sistema.
\item lpWindowName: representa o t�tulo da janela.
\item dwStyle: conjunto de flags que definem como a janela ser� exibida\cite{docwindowstyle}: com bordas, maximixada, minimizada, \ldots
\item x, y: posi��o inicial da janela.
\item nWidth, nHeight: dimens�es iniciais da janela.
\item hWndParent: se a janela que est� sendo criada � uma subjanela de uma j� existente, passa-se o handler da janela pai nesse argumento, caso contr�rio, passa-se NULL. Para a engine gr�fica, esse argumento sempre recebe o valor NULL.
\item hMenu: se for necess�rio criar uma menu, deve-se passar o handler do menu nesse argumento, caso contr�rio, passa-se NULL.
\item hInstance: o handler da inst�ncia do programa que est� sendo executado, esse par�metro � utilizado dentro do sistema na cria��o de um identificador �nico para a janela.
\item lpParam: dados do usu�rio. Esse par�metro recebe um ponteiro para qualquer tipo de dado que o usu�rio necessite. Na engine esse argumento recebe o ponteiro para uma inst�ncia de uma classe interna que representa a janela do sistema.
\end{itemize}

Para se registrar uma classe no sistema deve-se chamar a fun��o\cite{docregisterclass}:
\begin{verbatim}
ATOM WINAPI RegisterClass(
  __in  const WNDCLASS *lpWndClass
);
\end{verbatim}

O argumento lpWndClass um ponteiro para uma inst�ncia de \texttt{WNDCLASS}\cite{docwndclass}, uma estrutura que guarda, entre outras coisas, o nome da classe que est� sendo registrada, o ponteiro para a fun��o que ir� tratar os eventos gerados pelo sistema e uma flag que indica se o contexto do dispositivo\cite{docdevicecontext} pode ser compartilhado. Quando se cria um contexto do OpenGL � importante sempre registrar a classe como dona do contexto do dispositivo.

Ap�s criada, deve-se avisar o sistema que a janela � vis�vel, para isso, executa-se\cite{docshowwindow}:

\begin{verbatim}
BOOL WINAPI ShowWindow(
  __in  HWND hWnd,
  __in  int nCmdShow
);
\end{verbatim}

Onde hWnd � o handler retornado pela fun��o \texttt{CreateWindow} e nCmdShow controla como a janela deve ser exibida (maximizada, minimizada, simplesmente exibida, \ldots)

Depois de criar a janela, a thread m�e come�a a receber os eventos da janela criada em sua fila de mensagens. Para se recuperar uma mensagem da fila, utiliza-se a fun��o\cite{docgetmessage}:

\begin{verbatim}
BOOL WINAPI GetMessage(
  __out     LPMSG lpMsg,
  __in_opt  HWND hWnd,
  __in      UINT wMsgFilterMin,
  __in      UINT wMsgFilterMax
);
\end{verbatim}

Que recebe como par�metros um ponteiro para uma estrutura do tipo \texttt{MSG}\cite{docmsg}, o handler retornado pela fun��o \texttt{CreateWindow} e 

\subsubsection{O loop de eventos da janela}

A fun��o registrada na WNDCLASS que tratar� os eventos gerados pelo sistema deve ter a seguinte assinatura\cite{docwindowproc}:
\begin{verbatim}
LRESULT CALLBACK WindowProc(
  __in  HWND hwnd,
  __in  UINT uMsg,
  __in  WPARAM wParam,
  __in  LPARAM lParam
);
\end{verbatim}

\subsection{Transforma��es lineares}

\subsection{Tensores de imagens de resson�ncia magn�tica de difus�o}
Difus�o � o nome dado ao movimento aleat�rio de mol�culas em um fluido (l�quido ou gasoso). Denomina-se coeficiente de difus�o a facilidade com que uma mol�cula se move no meio. Em fluidos homog�neos, como a �gua, o coeficiente de difus�o � o mesmo em todas as dire��es. A esse tipo de meio d�-se o nome de isotr�pico. Em fluidos heterog�neos o coeficiente pode variar dependendo da dire��o. Esses s�o os meios anisotr�picos. Um exemplo de meio anisotr�pico s�o os tecidos biol�gicos.

Imagens de resson�ncia magn�tica de difus�o s�o um dos meios de obter informa��es sobre o coeficiente de difus�o de mol�culas de �gua em diferentes dire��es em tecidos biol�gicos.

Para representar tais coeficientes s�o utilizados tensores, que s�o abstra��es de escalares, vetores e matrizes utilizados em diversas aplica��es. Em imagens de resson�ncia magn�tica de difus�o, tais tensores s�o representados por matrizes $ 3 \times 3 $ sim�tricas.

% Pensar melhor nesse nome
\subsection{Representa��o gr�fica de tensores}
Em meios isotr�picos o coeficiente de difus�o pode ser representado por uma esfera, pois o movimento das mol�culas se distribui igualmente para todas as dire��es em um determinado per�odo de tempo. J� em meios anisotr�picos, como o movimento depende da dire��o, o coeficiente � modelado como um elips�ide, sendo que os semi-eixos t�m comprimento proporcional aos autovalores do tensor ($\lambda_{1} > \lambda_{2} > \lambda_{3}$) ao longo dos seus autovetores $\epsilon_{1}, \epsilon_{2}, \epsilon_{3}$.


\section{Atividades realizadas}
Id�ias aplicadas na solu��o dos problemas.

\subsection{Objetos da API gr�fica}

Nessa se��o ser�o descritos os mapeamentos utilizados pela engine para os objetos definidos pela API gr�fica.

\subsubsection {Shader}

\subsubsection {GPUProgram}

\subsubsection {Bufer}

\subsubsection {Texture1D}

\subsubsection {Texture2D}

\subsubsection {BufferTexture}

\subsection{Grafo de cena}

\subsubsection{Tipos de n�s}

\subsubsection{Node Visitors}

Todos os \texttt{Visitors} padr�o da engine herdam da classe abstrata \texttt{VisitorTrait}, que encapsula o algoritmo de busca em profundidade para grafos direcionados. 
Cada \texttt{visitor} � respons�vel por executar um determinado conjunto de m�todos durante sua travessia no grafo de cena.

%checar os nomes dos visitors pq eu realmente n�o me lembro....
Os \texttt{visitors} definidos pela engine s�o: \texttt{UpdatePassVisitor} e \texttt{ColorPassVisitor}.

O \texttt{UpdatePassVisitor} � respons�vel por fazer a travessia dentro do \texttt{UpdatePass} do renderizador. Durante a travessia esse visitor executa o m�todo \texttt{updatePass} e em seguida continua a busca em profundidade e, por fim, executa o m�todo \texttt{postUpdatePass}.

% Pensar melhor nesse nome
% Citar artigo da NVidia
Geometry instancing

\subsection{Renderizador}

O renderizador da engine gr�fica utiliza um algoritmo de 3 fases:

\begin{itemize}
\item Atualiza��o dos n�s do grafo de cena
\item Processamento do grafo de cena
\item P�s-Processamento da imagem gerada pela fase de processamento
\end{itemize}

\subsubsection {Fase de atualiza��o}

Nessa fase cada n� do grafo de cena � atualizado atrav�s de uma chamada ao m�todo updatePass da interface Node.

Para a implementa��o dessa fase foi utilizado o \texttt{Visitor}, % n�o esquecer de colocar referencia para o GOF aqui
que implementa uma busca em profundidade no grafo de cena. Essa � a �nica parte do algoritmo de renderiza��o que n�o pode ser customizada.

\subsubsection {Fase de processamento do grafo de cena}

Nessa fase executa-se uma sequ�ncia de algoritmos definida pelo usu�rio da engine, a execu��o � equivalente ao c�digo:

\begin{verbatim}
std::vector<SceneProcessor*>::iterator it;
for(it = processors.begin(); it != processors.end(); it++){
   if(it->isActive()){
       it->process(gfx, renderer); %verificar se os argumentos est�o corretos
   }
}
\end{verbatim}

Cada algoritmo deve implementar a interface \texttt{SceneProcessor}, que tem os seguintes m�todos:

\begin{itemize}
\item \texttt{bool isInitialized(GraphicAPI*)}: m�todo que indica se o algoritmo j� preparou todas as suas depend�ncias
\item \texttt{void initialize(GraphicAPI*, Renderer*)}: esse m�todo deve criar todas as dependencias do m�todo \texttt{process}
\item \texttt{void process(GraphicAPI*, Renderer*)}: executa o algoritmo de processamento
\item \texttt{bool isActive()}: indica se o algoritmo deve ou n�o ser executado
\end{itemize}

\subsubsection{Fase de p�s-processamento}

%Em algum lugar vai ser explicado o que � o teste de profundidade do fragmento?
Essa fase � semelhante � anterior, por�m o teste de profundidade do fragmento n�o � executado por padr�o, a implementa��o do algoritmo deve ativ�-lo manualmente.

A interface que deve ser implementada � a \texttt{ScenePostProcessor} que define m�todos com a mesma sem�ntica dos m�todos do \texttt{SceneProcessor}.

\subsubsection{Algoritmos de processamento de cena pr�-definidos}

A engine atualmente define apenas um algoritmo de processamento padr�o, o \texttt{ColorPass}. % checar o nome do algoritmo

Esse algoritmo executa uma busca em profundidade no grafo de cena executando o m�todo \texttt{colorPass} de cada n� do grafo de cena %Mudar se implementarmos o frustum culling

\subsubsection{Algoritmos de p�s-processamento de cena pr�-definidos}
 
Atualmente existem 2 algoritmos de p�s-processamento implementados pelas classes: \texttt{FramebufferImageProcessor} e \texttt{BlitToFramebuffer}.

O \texttt{FramebufferImageProcessor} implementa uma t�cnica de renderiza��o conhecida como ping-pong rendering % achar o termo t�cnico correto
% descrever a t�cnica e a implementa��o.

O \texttt{BlitToFramebuffer} renderiza um ret�ngulo com as dimens�es da tela com a textura armazenada no renderizador de nome \texttt{"color"}.
% blit to framebuffer

\subsection{Visualiza��o de campos tensoriais}


\section{Resultados e produtos obtidos}

O c�digo do Pandora's Box Graphics Engine, assim como da aplica��o de visualiza��o de campos tensoriais e outros exemplos pode ser encontrado no seguinte endere�o:

\url{https://github.com/victorkendy/PandoraBox}

\subsection{Compilador de campos tensoriais}
A aplica��o de compila��o de campos tensoriais pode ser executada atrav�s da linha de comando com os seguintes argumentos opcionais:

\begin{verbatim}
field_compiler.exe [ARQUIVO DO CAMPO] [ARQUIVO CTF]
\end{verbatim}

Caso o programa seja executado sem nenhum argumento o usu�rio deve escolher entre as op��es 1 e 2 (campo de uma dupla h�lice sint�tica e de um c�rebro humano respectivamente). A op��o \texttt{ARQUIVO DO CAMPO} deve indicar o nome de um arquivo .img sem a extens�o (sup�e-se que o arquivo .hdr tenha o mesmo nome), ou seja, suponha que existam os arquivos \texttt{campo.hdr} e \texttt{campo.img}, ent�o o argumento deve ser \texttt{campo}. O �ltimo argumento deve ser o nome do arquivo .ctf a ser criado, por exemplo \texttt{campo.ctf}.

\subsection{Visualizador de campos tensoriais}
O visualizador de campos tensoriais tamb�m pode ser executada atrav�s da linha de comando da seguinte maneira:

\begin{verbatim}
tensor_field.exe [ARQUIVO CTF]
\end{verbatim}

A execu��o sem nenhum argumento permitir� ao usu�rio escolher entre os dois campos utilizados como teste (dupla h�lice e c�rebro). � poss�vel visualizar outros campos no formato .ctf enviando como argumento o nome do arquivo, por exemplo \texttt{campo.ctf}.

\subsubsection{Movimenta��o e intera��o com o campo tensorial}
� poss�vel utilizar as seguintes teclas para interagir com o campo tensorial:

\begin{center}
	\begin{longtable}{ | c | p{10cm} | }
	\hline
	\multicolumn{2}{|p{11cm}|}{\textbf{Altern�ncia de efeitos}} \\ \hline
	1 & Ativa/desativa o inversor de cores \\ \hline
	2 & Ativa/desativa o filtro do componente vermelho das cores \\ \hline
	3 & Ativa/desativa o efeito de lentes senoidais \\ \hline
	4 & Ativa/desativa o depth peeling \\ \hline
	\multicolumn{2}{|p{11cm}|}{\textbf{Movimenta��o da c�mera}} \\ \hline
	W & Move a c�mera para cima \\ \hline
	S & Move a c�mera para baixo \\ \hline
	A & Move a c�mera para a esquerda \\ \hline
	D & Move a c�mera para a direita \\ \hline
	Q & Move a c�mera em dire��o ao campo \\ \hline
	E & Move a c�mera na dire��o oposta ao campo \\ \hline
	\multicolumn{2}{|p{11cm}|}{\textbf{N�vel de anisotropia considerado (s�o somente mostrados os elips�ides que representam tensores cuja anisotropia fracionada pertence ao intervalo [min\_AF,max\_AF])}} \\ \hline
	Z & Aumenta o valor de min\_AF \\ \hline
	X & Diminui o valor de min\_AF \\ \hline
	V & Aumenta o valor de max\_AF \\ \hline
	C & Diminui o valor de max\_AF \\ \hline
	\multicolumn{2}{|p{11cm}|}{\textbf{Escala dos elips�ides}} \\ \hline
	O & Diminui o tamanho dos elips�ides \\ \hline
	P & Aumenta o tamanho dos elips�ides \\ \hline
	\multicolumn{2}{|p{11cm}|}{\textbf{C�lculo da anisotropia fracionada}} \\ \hline
	R & Utiliza o c�lculo da forma linear \\ \hline
	T & Utiliza o c�lculo da forma planar \\ \hline
	Y & Utiliza o c�lculo da forma esf�rica \\ \hline
	U & Utiliza o c�lculo geral da anisotropia fracionada \\ \hline
	\multicolumn{2}{|p{11cm}|}{\textbf{Grade (eixos cartesianos)}} \\ \hline
	G & Mostra/esconde a grade \\ \hline
	\multicolumn{2}{|p{11cm}|}{\textbf{Rampa de cores}} \\ \hline
	F & Alterna entre as rampas de cores dispon�veis \\ \hline
	\end{longtable}
\end{center}

Al�m da intera��o com o teclado � poss�vel rotacionar o campo utilizando o mouse. O movimento se inicia com o clique do bot�o esquerdo do mouse e termina quando o bot�o � solto.

\subsection{Exemplo de c�digo}

A seguir ser� explicada a utiliza��o b�sica da Pandora's Box. O exemplo desenvolvido utiliza um modelo de uma esfera que � transformada em um elips�ide no vertex shader. Inicialmente � inclu�do o header da engine:

\begin{verbatim}
#include "pbge/pbge.h"
\end{verbatim}

Na fun��o \texttt{main} s�o definidas as configura��es da janela (t�tulo e dimens�es), associado um inicializador de cena customizado ao gerenciador de janelas e ent�o � enviado o sinal para a engine iniciar a renderiza��o das imagens na janela. � necess�rio que o usu�rio implemente um \texttt{SceneInitializer} que definir� a estrutura do grafo de cena.

\begin{verbatim}
int main() {
    pbge::Manager manager;
    MySceneInitializer sceneInitializer;
    manager.setWindowTitle("Ellipsoid demo");
    manager.setWindowDimensions(1024, 768);
    manager.setSceneInitializer(&sceneInitializer);
    manager.displayGraphics();
    return 0;
}
\end{verbatim}

O inicializador de cena deve herdar de \texttt{SceneInitializer} e implementar o operador \texttt{()}. � poss�vel ter acesso ao renderizador atrav�s da inst�ncia de \texttt{Window} recebida como argumento. S�o ent�o adicionados no renderizador os processadores de cena (\texttt{RenderPassProcessor}  e \texttt{BlitToFramebuffer}). Em seguida � criado o grafo de cena com um n� de transforma��o identidade como ra�z. O grafo � ent�o preenchido com outros n�s e devolvido.

\begin{verbatim}
class MySceneInitializer : public pbge::SceneInitializer {
public:
    pbge::SceneGraph * operator () (pbge::GraphicAPI * gfx, pbge::Window * window) {
        pbge::Renderer * renderer = window->getRenderer();
        renderer->addSceneProcessor(new pbge::RenderPassProcessor);
        renderer->addPostProcessor(new pbge::BlitToFramebuffer);
        pbge::Node * root = new pbge::TransformationNode;
        pbge::SceneGraph * graph = new pbge::SceneGraph(root);
        
        configureCamera(root, gfx);
        createModel(root, gfx);

        return graph;
    }

private:
    void configureCamera(pbge::Node * parent, pbge::GraphicAPI * gfx) {...}
    void createModel(pbge::Node * parent, pbge::GraphicAPI * gfx) {...}
};
\end{verbatim}

Para definir a posi��o da c�mera utiliza-se um n� de transforma��o linear que � adicionado como filho da ra�z. A c�mera � instanciada, configurada (dire��o e perspectiva) e adicionada aos filhos da transforma��o.

\begin{verbatim}
void configureCamera(pbge::Node * parent, pbge::GraphicAPI * gfx) {
    pbge::TransformationNode * cameraParent = 
                      pbge::TransformationNode::translation(0, 0, 10);
    pbge::CameraNode * camera = new pbge::CameraNode;
    camera->lookAt(math3d::vector4(0,1,0), math3d::vector4(0, 0, -1));
    camera->setPerspective(90, 1.0f, 2.0f, 30.0f);
    cameraParent->addChild(camera);
    parent->addChild(cameraParent);
}
\end{verbatim}

A esfera utilizada possui raio 2 e 100 subdivis�es. O n� de modelo (\texttt{ModelInstance}) � criado a partir do modelo da esfera. O shader criado � adicionado ao n�, que � ent�o adicionado ao grafo.

\begin{verbatim}
void createModel(pbge::Node * parent, pbge::GraphicAPI * gfx) {
    pbge::VBOModel * sphere = pbge::Geometrics::createSphere(2,100,gfx);
    pbge::ModelInstance * model = new pbge::ModelInstance(sphere);
    pbge::GPUProgram * shader = gfx->getFactory()->createProgramFromString(
        "#version 150\n"
        "in vec4 pbge_Vertex;\n"
        "out vec4 color\n;"
        "uniform mat4 pbge_ModelViewProjectionMatrix;\n"
        "void main() {\n"
        "   mat4 scale = mat4(1,0,0,0,\n"
        "                     0,2,0,0,\n"
        "                     0,0,1,0,\n"
        "                     0,0,0,1);\n"
        "   gl_Position = pbge_ModelViewProjectionMatrix*scale*pbge_Vertex;\n"
        "   color = vec4(pbge_Vertex.xyz, 1);\n"
        "}",
        "in vec4 color;\n"
        "void main() {\n"
        "   gl_FragColor = color;\n"
        "}"
        );
    model->setRenderPassProgram(shader);
    parent->addChild(model);
}
\end{verbatim}

O m�todo \texttt{createProgramFromString} recebe duas strings, sendo a primeira o c�digo do vertex shader e a segunda do fragment shader. Nesse exemplo o vertex shader definido recebe o v�rtice e a matriz ModelViewProjection e devolve a cor do fragmento. � criada uma matriz de escala que multiplica o v�rtice e o resultado � por sua vez multiplicado pela ModelViewProjection resultando na posi��o. A cor definida utiliza as coordenadas \texttt{x,y,z} como componentes \texttt{r,g,b}. O fragment shader define a cor do fragmento como a cor recebida do vertex shader.


\section{Conclus�es}
Para o desenvolvimento de aplica��es de computa��o gr�fica s�o necess�rias diversas t�cnicas de otimiza��o e organiza��o das informa��es. A Pandora's Box Graphics Engine tem por objetivo disponibilizar uma interface simples para utiliza��o de algumas dessas t�cnicas e de f�cil customiza��o:

\begin{itemize}
\item As cenas s�o representadas por grafos de cena nos quais � poss�vel utilizar os n�s j� definidos na engine ou criar novos tipos para implementar outras t�cnicas
\item Uma cena pode ser renderizada atrav�s dos processadores de cena adicionados ao renderizador sendo poss�vel tamb�m criar novos processadores (que percorram o grafo executando m�todos espec�ficos de n�s customizados, por exemplo)
\item A utiliza��o de shaders se resume a escrever o c�digo e associ�-lo a algum n� do grafo (a compila��o e o bind s�o gerenciados pela engine)
\item Texturas s�o gerenciadas pela engine, cabendo ao usu�rio carreg�-las e associ�-las a n�s do grafo de cena
\end{itemize}

Com o aux�lio da Pandora's Box foi poss�vel desenvolver algumas aplica��es de exemplo, incluindo o visualizador de campos tensoriais de imagens de resson�ncia magn�tica sens�veis � difus�o. Essa aplica��o exigiu a utiliza��o de diversas t�cnicas de otimiza��o para ser poss�vel a visualiza��o de campos com grandes quantidades de tensores em tempo real.

% Falta escrever mais um pouco e fechar tudo


\nocite{*}
\newpage \bibliographystyle{plain}
\bibliography{bibliografia_monografia}

\end{document}
