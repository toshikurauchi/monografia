\section{Resultados e produtos obtidos}

O c�digo do Pandora's Box Graphics Engine, assim como da aplica��o de visualiza��o de campos tensoriais e outros exemplos pode ser encontrado no seguinte endere�o:

\url{https://github.com/victorkendy/PandoraBox}

\subsection{Compilador de campos tensoriais}
A aplica��o de compila��o de campos tensoriais pode ser executada atrav�s da linha de comando com os seguintes argumentos opcionais:

\begin{verbatim}
field_compiler.exe [ARQUIVO DO CAMPO] [ARQUIVO CTF]
\end{verbatim}

Caso o programa seja executado sem nenhum argumento o usu�rio deve escolher entre as op��es 1 e 2 (campo de uma dupla h�lice sint�tica e de um c�rebro humano respectivamente). A op��o \texttt{ARQUIVO DO CAMPO} deve indicar o nome de um arquivo .img sem a extens�o (sup�e-se que o arquivo .hdr tenha o mesmo nome), ou seja, suponha que existam os arquivos \texttt{campo.hdr} e \texttt{campo.img}, ent�o o argumento deve ser \texttt{campo}. O �ltimo argumento deve ser o nome do arquivo .ctf a ser criado, por exemplo \texttt{campo.ctf}.
