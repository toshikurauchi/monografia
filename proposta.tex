\documentclass[12pt]{article}
\usepackage[brazil]{babel}
\usepackage{setspace}
\usepackage{boxedminipage}
\usepackage[latin1]{inputenc}

\setlength{\paperheight}{29.7cm}
\setlength{\footskip}{1.0cm}
\setlength{\textheight}{20.0cm}
\setlength{\textwidth}{16.0cm}
\setlength{\oddsidemargin}{0.0cm}
\setlength{\topmargin}{0.0cm}


\begin{document}

\title{PandoraBox Graphic Engine - uma engine gr�fica otimizada para visualiza��o de campos tensoriais}

\author{Andrew Toshiaki Nakayama Kurauchi \\ 5894035 \and Victor Kendy Harada \\ 5893823}

\maketitle
\thispagestyle{empty}

\vfill
\begin{flushright}
\emph{Supervisor:} \\
Prof. Doutor Marcel Parolin Jackowski
\end{flushright}

\newpage
\setcounter{page}{1}
	
\section{Introdu��o}
	Introdu��o

\section{Resumo da monografia}
	
	O estudo de tensores e campos tensoriais � de import�ncia fundamental para diversas �reas do conhecimento. S�o encontradas aplica��es 

	O desenvolvimento de programas de computa��o gr�fica voltados � visualiza��o cient�fica em tempo real exige a aplica��o de um vasto conjunto de t�cnicas de otimiza��o. 

\section{Objetivos do trabalho}
	Objetivos
	
\section{Atividades j� realizadas}

	As seguintes atividades j� foram desenvolvidas:
	
	\begin{itemize}
		\item Estudo inicial sobre tensores e visualiza��o de campos tensoriais
		\item Codifica��o da estrutura geral da engine:
			\begin{itemize}
				\item Renderizador
				\item Grafo de cena
				\item Gerenciador de janelas
				\item C�mera
				\item Transforma��es lineares
				\item Modelos
				\item Luzes
				\item Suporte a shaders customizados
			\end{itemize}
		\item Leitura de campos tensoriais representados em arquivos no formato Analyze
		\item C�lculo dos autovalores e autovetores dos tensores e renderiza��o dos modelos de elips�ides
	\end{itemize}
	
\section{Cronograma de Atividades}

	Cronograma
	
\section{Estrutura esperada da monografia}
	
	Estrutura

\end{document}
